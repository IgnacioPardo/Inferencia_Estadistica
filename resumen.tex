% Options for packages loaded elsewhere
\PassOptionsToPackage{unicode}{hyperref}
\PassOptionsToPackage{hyphens}{url}
%
\documentclass[
]{article}
\usepackage{amsmath,amssymb}
\usepackage{lmodern}
\usepackage{iftex}
\ifPDFTeX
  \usepackage[T1]{fontenc}
  \usepackage[utf8]{inputenc}
  \usepackage{textcomp} % provide euro and other symbols
\else % if luatex or xetex
  \usepackage{unicode-math}
  \defaultfontfeatures{Scale=MatchLowercase}
  \defaultfontfeatures[\rmfamily]{Ligatures=TeX,Scale=1}
\fi
% Use upquote if available, for straight quotes in verbatim environments
\IfFileExists{upquote.sty}{\usepackage{upquote}}{}
\IfFileExists{microtype.sty}{% use microtype if available
  \usepackage[]{microtype}
  \UseMicrotypeSet[protrusion]{basicmath} % disable protrusion for tt fonts
}{}
\makeatletter
\@ifundefined{KOMAClassName}{% if non-KOMA class
  \IfFileExists{parskip.sty}{%
    \usepackage{parskip}
  }{% else
    \setlength{\parindent}{0pt}
    \setlength{\parskip}{6pt plus 2pt minus 1pt}}
}{% if KOMA class
  \KOMAoptions{parskip=half}}
\makeatother
\usepackage{xcolor}
\usepackage[margin=1in]{geometry}
\usepackage{graphicx}
\makeatletter
\def\maxwidth{\ifdim\Gin@nat@width>\linewidth\linewidth\else\Gin@nat@width\fi}
\def\maxheight{\ifdim\Gin@nat@height>\textheight\textheight\else\Gin@nat@height\fi}
\makeatother
% Scale images if necessary, so that they will not overflow the page
% margins by default, and it is still possible to overwrite the defaults
% using explicit options in \includegraphics[width, height, ...]{}
\setkeys{Gin}{width=\maxwidth,height=\maxheight,keepaspectratio}
% Set default figure placement to htbp
\makeatletter
\def\fps@figure{htbp}
\makeatother
\setlength{\emergencystretch}{3em} % prevent overfull lines
\providecommand{\tightlist}{%
  \setlength{\itemsep}{0pt}\setlength{\parskip}{0pt}}
\setcounter{secnumdepth}{-\maxdimen} % remove section numbering
\ifLuaTeX
  \usepackage{selnolig}  % disable illegal ligatures
\fi
\IfFileExists{bookmark.sty}{\usepackage{bookmark}}{\usepackage{hyperref}}
\IfFileExists{xurl.sty}{\usepackage{xurl}}{} % add URL line breaks if available
\urlstyle{same} % disable monospaced font for URLs
\hypersetup{
  pdftitle={Resumen Primer Parcial Inferencia Estadística},
  pdfauthor={Ignacio Pardo},
  hidelinks,
  pdfcreator={LaTeX via pandoc}}

\title{Resumen Primer Parcial Inferencia Estadística}
\author{Ignacio Pardo}
\date{}

\begin{document}
\maketitle

\hypertarget{resumen-primer-parcial-inferencia-estaduxedstica}{%
\section{Resumen Primer Parcial Inferencia
Estadística}\label{resumen-primer-parcial-inferencia-estaduxedstica}}

\hypertarget{introducciuxf3n}{%
\subsection{Introducción}\label{introducciuxf3n}}

Resumen de la materia de inferencia estadística de la Licenciatura en
Tecnología Digital en la Universidad Torcuato Di Tella.

\hypertarget{contenido}{%
\subsection{Contenido}\label{contenido}}

\begin{itemize}
\tightlist
\item
  \protect\hyperlink{resumen-primer-parcial-inferencia-estaduxedstica}{Resumen
  Primer Parcial Inferencia Estadística}

  \begin{itemize}
  \tightlist
  \item
    \protect\hyperlink{introducciuxf3n}{Introducción}
  \item
    \protect\hyperlink{contenido}{Contenido}
  \item
    \protect\hyperlink{esperanza}{Esperanza}
  \item
    \protect\hyperlink{varianza}{Varianza}

    \begin{itemize}
    \tightlist
    \item
      \protect\hyperlink{desvuxedo-estuxe1ndar}{Desvío Estándar}
    \item
      \protect\hyperlink{covarianza}{Covarianza}
    \item
      \protect\hyperlink{correlaciuxf3n}{Correlación}
    \end{itemize}
  \item
    \protect\hyperlink{continuas}{Continuas}

    \begin{itemize}
    \tightlist
    \item
      \protect\hyperlink{distribuciuxf3n-normal}{Distribución Normal}

      \begin{itemize}
      \tightlist
      \item
        \protect\hyperlink{funciuxf3n-acumulada}{Función acumulada}
      \end{itemize}
    \item
      \protect\hyperlink{distribuciuxf3n-uniforme}{Distribución
      Uniforme}
    \item
      \protect\hyperlink{distribuciuxf3n-exponencial}{Distribución
      Exponencial}
    \end{itemize}
  \item
    \protect\hyperlink{discretas}{Discretas}

    \begin{itemize}
    \tightlist
    \item
      \protect\hyperlink{distribuciuxf3n-bernoulli}{Distribución
      Bernoulli}
    \item
      \protect\hyperlink{distribuciuxf3n-binomial}{Distribución
      Binomial}
    \item
      \protect\hyperlink{distribuciuxf3n-poisson}{Distribución Poisson}
    \end{itemize}
  \item
    \protect\hyperlink{convergencia-en-probabilidad}{Convergencia en
    Probabilidad}

    \begin{itemize}
    \tightlist
    \item
      \protect\hyperlink{propiedades}{Propiedades}
    \end{itemize}
  \item
    \protect\hyperlink{estimaciuxf3n-por-lgn}{Estimación por LGN}

    \begin{itemize}
    \tightlist
    \item
      \protect\hyperlink{estimaciuxf3n-esperanza}{Estimación Esperanza}
    \item
      \protect\hyperlink{estimaciuxf3n-varianza}{Estimación Varianza}
    \item
      \protect\hyperlink{estimaciuxf3n-proporciuxf3n}{Estimación
      Proporción}
    \item
      \protect\hyperlink{estimaciuxf3n-de-probabilidad}{Estimación de
      Probabilidad}
    \end{itemize}
  \item
    \protect\hyperlink{formulas-consistencia}{Formulas Consistencia}

    \begin{itemize}
    \tightlist
    \item
      \protect\hyperlink{sesgo}{Sesgo}

      \begin{itemize}
      \tightlist
      \item
        \protect\hyperlink{asintuxf3ticamente-insesgado}{Asintóticamente
        Insesgado}
      \end{itemize}
    \item
      \protect\hyperlink{error-estuxe1ndar}{Error Estándar}
    \item
      \protect\hyperlink{error-cuadruxe1tico-medio}{Error Cuadrático
      Medio}
    \end{itemize}
  \item
    \protect\hyperlink{desigualdad-de-chebyshev}{Desigualdad de
    Chebyshev}
  \item
    \protect\hyperlink{desigualdad-de-markov}{Desigualdad de Markov}
  \item
    \protect\hyperlink{momentos}{Momentos}

    \begin{itemize}
    \tightlist
    \item
      \protect\hyperlink{momentos-de-una-variable-aleatoria}{Momentos de
      una variable aleatoria}

      \begin{itemize}
      \tightlist
      \item
        \protect\hyperlink{discreta}{Discreta}
      \item
        \protect\hyperlink{continua}{Continua}
      \end{itemize}
    \end{itemize}
  \item
    \protect\hyperlink{estimaciuxf3n-por-muxe1xima-verosimilitud-likelihood}{Estimación
    por Máxima Verosimilitud (Likelihood)}

    \begin{itemize}
    \tightlist
    \item
      \protect\hyperlink{log-likelihood}{Log-likelihood}

      \begin{itemize}
      \tightlist
      \item
        \protect\hyperlink{optimizaciuxf3n}{Optimización}
      \end{itemize}
    \end{itemize}
  \item
    \protect\hyperlink{intervalos-de-confianza}{Intervalos de Confianza}

    \begin{itemize}
    \tightlist
    \item
      \protect\hyperlink{intervalo-de-confianza-para-mu}{Intervalo de
      Confianza para \(\mu\)}
    \item
      \protect\hyperlink{t-student}{T-Student}
    \end{itemize}
  \end{itemize}
\end{itemize}

\hypertarget{esperanza}{%
\subsection{Esperanza}\label{esperanza}}

\[
\begin{align*}
    {\operatorname{E}(X)=\sum_{i=1}^{n}x_{i}\operatorname{F_x}(x_i)} \\
    \operatorname{E}(\overline{X}_n)= \frac{1}{n}\times\sum_{i=1}^{n}{\operatorname {E}(X_i)}
\end{align*}
\]

Si \(X\) y \(Y\) son variables aleatorias con esperanza finita y
\({a,b,c \in \mathbb {R} }\) son constantes entonces

\begin{itemize}
\tightlist
\item
  \({\operatorname {E} [c]=c}\)
\item
  \({\operatorname {E} [cX]=c\operatorname {E} [X]}\)
\item
  \({\operatorname {E} [X+Y]=\operatorname {E} [X]+\operatorname {E} [Y]}\)
\item
  Si \({X\geq 0}\) entonces \({\operatorname {E} [X]\geq 0}\)
\item
  Si \({X\leq Y}\) entonces
  \({\operatorname {E} [X]\leq \operatorname {E} [Y]}\)
\item
  Si \(X\) está delimitada por dos números reales, \(a\) y \(b\), esto
  es \({a<X<b}\) entonces también lo está su media, es decir,
  \({a<\operatorname {E} [X]<b}\)
\item
  Si \({Y=a+bX}\), entonces
  \({\operatorname {E} [Y]=\operatorname {E} [a+bX]=a+b\operatorname {E} [X]}\)
\item
  Si \(X\) y \(Y\) son variables aleatorias independientes entonces
  \({\operatorname {E} [XY]=\operatorname {E} [X]\operatorname {E} [Y]}\)
\end{itemize}

\hypertarget{varianza}{%
\subsection{Varianza}\label{varianza}}

\(\operatorname {Var}[X]=\operatorname {E} [X^{2}]-\operatorname {E} [X]^{2} \implies \operatorname {E} [X^{2}] = \operatorname {Var}[X] + \operatorname {E} [X]^{2}\)

Sean \(X\) y \(Y\) dos variables aleatorias con varianza finita y
\({a\in \mathbb {R} }\)

\begin{itemize}
\tightlist
\item
  \({\operatorname {Var} (X)\geq 0}\)
\item
  \({\operatorname {Var} (a)=0}\)
\item
  \({\operatorname {Var} (aX)=a^{2}\operatorname {Var} (X)}\)
\item
  \({\operatorname {Var} (X+Y)=\operatorname {Var} (X)+\operatorname {Var} (Y)+2\operatorname {Cov} (X,Y)}\),
  donde \({\operatorname {Cov} (X,Y)}\) denota la covarianza de \(X\) e
  \(Y\)
\item
  \({\operatorname {Var} (X+Y)=\operatorname {Var} (X)+\operatorname {Var} (Y)}\)
  si \(X\) y \(Y\) son variables aleatorias independientes.
\item
  \({\operatorname {Var} (Y)=\operatorname {E} (\operatorname {Var} (Y|X))+\operatorname {Var} (\operatorname {E} (Y|X))}\)
  cálculo de la Varianza por Pitágoras, dónde \({Y|X}\) es la variable
  aleatoria condicional \(Y\) dado \(X\).
\end{itemize}

\hypertarget{desvuxedo-estuxe1ndar}{%
\subsubsection{Desvío Estándar}\label{desvuxedo-estuxe1ndar}}

\({\operatorname{SD}(X) = \sigma ={\sqrt {{\text{Var}}(X)}}} \implies{\sigma ^{2}={\text{Var}}(X)}\)

\hypertarget{covarianza}{%
\subsubsection{Covarianza}\label{covarianza}}

\({\operatorname {Cov} (X,Y)=\operatorname {E} \left[XY\right]-\operatorname {E} \left[X\right]\operatorname {E} \left[Y\right]}\)

\hypertarget{correlaciuxf3n}{%
\subsubsection{Correlación}\label{correlaciuxf3n}}

\(\rho_{xy} = {\frac{\operatorname{cov}_{xy}}{\sigma_x\sigma_y}} = {\frac{\operatorname{cov}_{xy}}{\operatorname{SD}(x)\operatorname{SD}(y)}}\)

\hypertarget{continuas}{%
\subsection{Continuas}\label{continuas}}

\[
{\operatorname{F_x}(x) = \int_{-\infty}^{x}{f_X(u)du}} \\
f_X(x) = \frac{d}{dx}{\operatorname{F_x}(x)} \\
\]

\[{\operatorname{P}(a < X < b) = \int_a^b{f(x)dx} = \operatorname{F_x}(b) -\operatorname{F_x}(a)} \]

\[{\operatorname{P}(a < X < b) = \operatorname{P}(a < X \leq b) = \operatorname{P}(a \leq X < b)= \operatorname{P}(a \leq X \leq b)}\]

\hypertarget{distribuciuxf3n-normal}{%
\subsubsection{Distribución Normal}\label{distribuciuxf3n-normal}}

Si \({X\sim N(\mu ,\sigma ^{2})}\) y \({a,b\in \mathbb {R} }\), entonces
\({aX+b\sim N(a\mu +b,a^{2}\sigma ^{2})}\)

Si \({X\,\sim N(\mu ,\sigma ^{2})\,}\), entonces
\({Z={\frac {X-\mu }{\sigma }}\!}\) es una variable aleatoria normal
estándar: \(Z\) \textasciitilde{} \(N(0,1)\).

\[{X\,\sim N(\mu, \sigma ^{2}) \implies Z={\frac {X-\mu }{\sigma }} \sim N(0,1)}\]

\[{Z\sim N(0,1) \implies X = \sigma Z + \mu \sim N(\mu, \sigma ^{2})}\]

\[
\begin{align}
    X & \sim N(\mu, \sigma^2) \\
    X+b & \sim N(\mu+b, \sigma^2) \\
    aX & \sim N(a \times \mu, a^2 \times \sigma^2) \\
    \frac{X - \mu}{\sigma} & \sim N(0, 1) \\
    \overline{X}_n & \sim N(\mu, \sigma^2/n) \text{, si } X_i \text{ son i.i.d}\\
    \frac{\overline{X}_n - \mu}{\sigma/\sqrt{n}} & \sim N(0, 1) \text{, si } X_i \text{ son i.i.d} \\
\end{align}
\]

\hypertarget{funciuxf3n-acumulada}{%
\paragraph{Función acumulada}\label{funciuxf3n-acumulada}}

\[
    \operatorname{f}(x) = \operatorname{\phi}(x) = \frac{1}{\sqrt{2 \times \pi \times \sigma^2}} \times e^{-\frac{(x-\mu)^2}{2 \times \sigma^2}}
\]

\hypertarget{distribuciuxf3n-uniforme}{%
\subsubsection{Distribución Uniforme}\label{distribuciuxf3n-uniforme}}

\[
\begin{align}
    \operatorname{P} (a<X<b) & =\frac {1}{b-a} \\\
    \operatorname{E}(X) & = \frac{a+b}{2} \\
    \operatorname{Var}(X) & = \frac{(b-a)^2}{12}
\end{align}
\]

\hypertarget{distribuciuxf3n-exponencial}{%
\subsubsection{Distribución
Exponencial}\label{distribuciuxf3n-exponencial}}

\[
\begin{align}
    \operatorname{f_X}(X) & = \lambda e^{-\lambda X} \text{, para } X \geq 0 \\
    \operatorname{F_X}(x) & = \operatorname{P} (X>x)=1-e^{-\lambda x} \\\
    \operatorname{E}(X) & = \frac{1}{\lambda} \\
    \operatorname{Var}(X) & = \frac{1}{\lambda^2}
\end{align}
\]

\hypertarget{discretas}{%
\subsection{Discretas}\label{discretas}}

\hypertarget{distribuciuxf3n-bernoulli}{%
\subsubsection{Distribución Bernoulli}\label{distribuciuxf3n-bernoulli}}

\[
\begin{align}
    \operatorname{P}(X=1) & = p \\
    \operatorname{P}(X=0) & = 1-p \\
    \operatorname{E}(X) & = p \\
    \operatorname{Var}(X) & = p(1-p)
\end{align}
\]

\hypertarget{distribuciuxf3n-binomial}{%
\subsubsection{Distribución Binomial}\label{distribuciuxf3n-binomial}}

\[
\begin{align}
    {\operatorname {P} (X=k) & =\binom {n}{k}p^{k}(1-p)^{n-k}} \\
    {\operatorname {E} (X) & =np} \\
    {\operatorname {Var} (X) & =np(1-p)}
\begin{align}
\]

\hypertarget{distribuciuxf3n-poisson}{%
\subsubsection{Distribución Poisson}\label{distribuciuxf3n-poisson}}

\[
\begin{align}
    \operatorname{P}(X=k) & = \frac{\lambda^k e^{-\lambda}}{k!} \\
    \operatorname{E}(X) & = \lambda \\
    \operatorname{Var}(X) & = \lambda
\begin{align}
\]

\hypertarget{convergencia-en-probabilidad}{%
\subsection{Convergencia en
Probabilidad}\label{convergencia-en-probabilidad}}

Sean \(X_{n}\) una secuencia de variables aleatorias,
\(X_{n}\xrightarrow{p} X\) si \(\forall \epsilon > 0\)

\(\lim _{n\rightarrow \infty }\operatorname {P} \left( |\overline{X}_{n} - \operatorname{E}(X) | > \epsilon \right) =0\),
por Ley de los Grandes Números.

\hypertarget{propiedades}{%
\subsubsection{Propiedades}\label{propiedades}}

Si \(X_{n}\xrightarrow{p} a\) y \(Y_{n}\xrightarrow{p} b\), entonces:

\begin{itemize}
\tightlist
\item
  \(X_{n}+Y_{n}\xrightarrow{p} a+b\)
\item
  \(X_{n}Y_{n}\xrightarrow{p} a \cdot b\)
\item
  \(\frac{X_{n}}{Y_{n}}\xrightarrow{p} \frac{a}{b}\) si \(b \neq 0\)
\item
  \(\operatorname{g}(X_{n})\xrightarrow{p} \operatorname{g}(a)\) si
  \(\operatorname{g}\) es una función continua
\end{itemize}

\hypertarget{estimaciuxf3n-por-lgn}{%
\subsection{Estimación por LGN}\label{estimaciuxf3n-por-lgn}}

Sean \(X_{1},X_{2},\dots ,X_{n}\) variables aleatorias independientes e
idénticamente distribuidas (i.i.d.) con esperanza \(\mu\) y varianza
\(\sigma^{2}\). Entonces, la media muestral
\(\overline{X}_{n} \xrightarrow{p} \mu\) por LGN. El estimador
\(\hat{\mu}_{n}=\overline{X}_{n}\) es consistente.

\hypertarget{estimaciuxf3n-esperanza}{%
\subsubsection{Estimación Esperanza}\label{estimaciuxf3n-esperanza}}

\[
\begin{align*}
    & \text{parámetro de interés: } \mu = \operatorname{E}(X) \\
    & \text{muestra aleatoria: } X_1, \dots, X_n \sim f \text{, i.i.d}\\
    & \text{estimador: } \hat{\mu}_n = \overline{X}_n = \frac{1}{n} \sum_{i=1}^n X_i \\
    & \text{estimador consistente: } \hat{\mu}_n = \overline{X}_{n}\xrightarrow{p} \mu
\end{align*}
\]

\hypertarget{estimaciuxf3n-varianza}{%
\subsubsection{Estimación Varianza}\label{estimaciuxf3n-varianza}}

\[
\begin{align*}
    & \text{parámetro de interés: } \sigma^2 = \operatorname{Var}(X) \\
    & \text{muestra aleatoria: } X_1, \dots, X_n \sim f \text{, i.i.d}\\
    & \text{estimador: } \hat{\sigma}^2_n = \frac{1}{n} \sum_{i=1}^n (X_i - \overline{X}_n)^2 \\
    & \text{estimador: } \hat{s}^2_n = \frac{1}{n-1} \sum_{i=1}^n (X_i - \overline{X}_n)^2
\end{align*}
\]

\hypertarget{estimaciuxf3n-proporciuxf3n}{%
\subsubsection{Estimación
Proporción}\label{estimaciuxf3n-proporciuxf3n}}

\[
\begin{align*}
    & \text{parámetro de interés: } p = \operatorname{P}(X=1) \\
    & \text{muestra aleatoria: } X_1, \dots, X_n \sim f \text{, i.i.d}\\
    & \text{estimador: } \hat{p}_n = \frac{1}{n} \sum_{i=1}^n X_i \\
    & \text{estimador consistente: } \overline{X}_{n}\xrightarrow{p} p \text{, por LGN}
\end{align*}
\]

\hypertarget{estimaciuxf3n-de-probabilidad}{%
\subsubsection{Estimación de
Probabilidad}\label{estimaciuxf3n-de-probabilidad}}

\[
\begin{align*}
    & \text{parámetro de interés: } p = \operatorname{F}(x) = \operatorname{P}(X \leq x) \\
    & \text{muestra aleatoria: } X_1, \dots, X_n \sim f \text{, i.i.d} \\
    & \text{definimos } Y_i \sim \operatorname{Bernoulli}(p) \\
    & Y_i = X_i \leq x = \{1 \text{ si } X_i \leq x \text{, } 0 \text{ si } X_i > x \\
    & \text{estimador: } \hat{F}_n(x) = \overline{Y}_{n} = \frac{1}{n} \sum_{i=1}^n \mathbb{1}_{X_i \leq x} \\
    & \text{estimador consistente: }\hat{F}_n(x) = \overline{Y}_{n} \xrightarrow{p} \operatorname{F}(x)
\end{align*}
\]

\hypertarget{formulas-consistencia}{%
\subsection{Formulas Consistencia}\label{formulas-consistencia}}

\hypertarget{sesgo}{%
\subsubsection{Sesgo}\label{sesgo}}

\[
  \operatorname{Sesgo}(\hat{\theta}\_n) = \operatorname{E}(\hat{\theta}\_n) - \theta 
\]

Si
\(\operatorname{Sesgo}(\hat{\theta}_n) = 0 \implies \operatorname{E}(\hat{\theta}_n) = \theta\)
entonces \(\hat{\theta}_n\) es insesgado.

\hypertarget{asintuxf3ticamente-insesgado}{%
\paragraph{Asintóticamente
Insesgado}\label{asintuxf3ticamente-insesgado}}

\[
    \operatorname{Sesgo}(\hat{\theta}_n) = \operatorname{E}(\hat{\theta}_n) - \theta \xrightarrow{p} 0
\]

\hypertarget{error-estuxe1ndar}{%
\subsubsection{Error Estándar}\label{error-estuxe1ndar}}

\[
\begin{align*}
    & \operatorname{SE}(\hat{\theta}_n) = \sqrt{\operatorname{Var}(\hat{\theta}_n)}\\
    & \operatorname{Var}(\hat{\theta}_n) = \operatorname{E}[(\hat{\theta}_n - \operatorname{E}(\hat{\theta}_n))^2]
\end{align*}
\]

\hypertarget{error-cuadruxe1tico-medio}{%
\subsubsection{Error Cuadrático Medio}\label{error-cuadruxe1tico-medio}}

\[
\begin{align*}
    & \operatorname{ECME}(\hat{\theta}_n) = \operatorname{Sesgo}(\hat{\theta}_n)^2 + \operatorname{Var}(\hat{\theta}_n) \\
    & \operatorname{ECME}(\hat{\theta}_n) = \operatorname{E}[(\hat{\theta}_n - \theta)^2]
\end{align*}
\]

\hypertarget{desigualdad-de-chebyshev}{%
\subsection{Desigualdad de Chebyshev}\label{desigualdad-de-chebyshev}}

\[
    \operatorname{P}(|X - \mu| \geq \epsilon) \leq \frac{\operatorname{Var}(X)}{\epsilon^2}
\]

\hypertarget{desigualdad-de-markov}{%
\subsection{Desigualdad de Markov}\label{desigualdad-de-markov}}

\[
    \operatorname{P}(X > \epsilon) \leq \frac{\operatorname{E}(X)}{\epsilon}
\]

\hypertarget{momentos}{%
\subsection{Momentos}\label{momentos}}

\hypertarget{momentos-de-una-variable-aleatoria}{%
\subsubsection{Momentos de una variable
aleatoria}\label{momentos-de-una-variable-aleatoria}}

\hypertarget{discreta}{%
\paragraph{Discreta}\label{discreta}}

\[
    m_{k} = \operatorname{E}(X^{k}) = \sum_{i=1}^{n}{(x_i - \overline{x}) ^ k}
\]

\hypertarget{continua}{%
\paragraph{Continua}\label{continua}}

\[
    m_{k} = \int_{\mathbb{R}} x^k f(x) dx
\]

\[
\begin{align*}
    & \operatorname{E}(X) = \mu \\
    & \operatorname{E}(X^2) = \mu^2 + \sigma^2 \\
    & \operatorname{E}(X^3) = \mu^3 + 3\mu\sigma^2 \\
    & \operatorname{E}(X^4) = \mu^4 + 6\mu^2\sigma^2 + 3\sigma^4
\end{align}
\]

\hypertarget{estimaciuxf3n-por-muxe1xima-verosimilitud-likelihood}{%
\subsection{Estimación por Máxima Verosimilitud
(Likelihood)}\label{estimaciuxf3n-por-muxe1xima-verosimilitud-likelihood}}

\[
    \mathcal{L}(\theta; \underline{X}) = \prod_{i=1}^n f(\theta; x_i)
\]

\hypertarget{log-likelihood}{%
\subsubsection{Log-likelihood}\label{log-likelihood}}

\[
    \ell(\theta; \underline{X}) = \ln(\mathcal{L}(\theta; \underline{X})) = \sum_{i=1}^n \ln(f(\theta; x_i))
\]

\hypertarget{optimizaciuxf3n}{%
\paragraph{Optimización}\label{optimizaciuxf3n}}

\[
    \frac{d\ell(\theta; \underline{X})}{d\theta} = \frac{d\ln(\mathcal{L}(\theta; \underline{X}))}{d\theta} = \sum_{i=1}^n \frac{d\ln(f(\theta; x_i))}{d\theta} = 0
\]

\hypertarget{intervalos-de-confianza}{%
\subsection{Intervalos de Confianza}\label{intervalos-de-confianza}}

\hypertarget{intervalo-de-confianza-para-mu}{%
\subsubsection{\texorpdfstring{Intervalo de Confianza para
\(\mu\)}{Intervalo de Confianza para \textbackslash mu}}\label{intervalo-de-confianza-para-mu}}

Sea la Variable Aleatoria \(X_i \sim \operatorname{N}(\mu, \sigma^2)\),
nuestro parámetro de interés es \(\mu\).
\(\overline{X}_n \sim \operatorname{N}(\mu, \sigma^2/n)\). Si
\(\sigma^2\) es conocido, entonces \(\overline{X}_n\) es insesgado y su
varianza es \(\sigma^2/n\).

Sea
\(Z = \frac{\overline{X}_{n} - \mu}{\sigma/\sqrt{n}} \sim \operatorname{N}(0, 1)\),
\$\hat{\mu}\emph{n =} \overline{X}{n} \$

El intervalo de confianza \(1-\alpha\) es:

\[
\begin{align*}
    \operatorname{P}(-z \leq Z \leq z) & = 1 - \alpha \\
    \phi(z) =  \operatorname{P}(Z \leq z) & = 1 - \alpha/2 \\
    \operatorname{P}(-z_{\alpha/2} \leq \frac{\overline{X}_{n} - \mu}{\sigma/\sqrt{n}} \leq z_{\alpha/2}) & = 1 - \alpha \\
    P(\hat{\mu}_n - z_{\alpha/2} \frac{\hat{\sigma}}{\sqrt{n}} \leq \mu \leq \hat{\mu}_n + z_{\alpha/2} \frac{\hat{\sigma}}{\sqrt{n}}) & = 1 - \alpha \\
    \text{IC = }(\hat{\mu}_n - z_{\alpha/2} \frac{\hat{\sigma}}{\sqrt{n}} \text{, } \hat{\mu}_n + z_{\alpha/2} \frac{\hat{\sigma}}{\sqrt{n}}) \\
    \overline{X}_{n} \pm z_{\alpha/2} \frac{\hat{\sigma}}{\sqrt{n}}
\end{align}
\]

\hypertarget{t-student}{%
\subsubsection{T-Student}\label{t-student}}

Si \(\sigma^2\) es desconocido, entonces \(\overline{X}_n\) es insesgado
y su varianza es \(s^2/n\)

Sea
\(T = \frac{\overline{X}_{n} - \mu}{\sqrt{\frac{s^2}{n}}} = \frac{\overline{X}_{n} - \mu}{s/\sqrt{n}} \sim \operatorname{t}_{n-1}\),
\$\hat{\mu}\emph{n =} \overline{X}{n} \$

El intervalo de confianza \(1-\alpha\) es:

\[
\begin{align*}
    \operatorname{P}(-t_{n-1,\alpha/2} \leq T \leq t_{n-1,\alpha/2}) & = 1 - \alpha \\
    \operatorname{P}(\hat{\mu}_n - t_{n-1,\alpha/2} \sqrt{\frac{s^2}{n}} \leq \mu \leq \hat{\mu}_n + t_{n-1,\alpha/2} \sqrt{\frac{s^2}{n}}) & = 1 - \alpha \\
    \operatorname{P}(\hat{\mu}_n - t_{n-1,\alpha/2} \frac{s}{\sqrt{n}} \leq \mu \leq \hat{\mu}_n + t_{n-1,\alpha/2} \frac{s}{\sqrt{n}}) & = 1 - \alpha \\
    \text{IC = }(\hat{\mu}_n - t_{n-1,\alpha/2} \frac{s}{\sqrt{n}} \text{, } \hat{\mu}_n + t_{n-1,\alpha/2} \frac{s}{\sqrt{n}}) \\
    \overline{X}_{n} \pm t_{n-1,\alpha/2} \frac{s}{\sqrt{n}}
\end{align*}
\]

\end{document}
